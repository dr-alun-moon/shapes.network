%& pdflatex
\documentclass{ltxdoc}
\usepackage[british]{babel}

\title{Network Diagram Shapes Library}
\author{Dr Alun Moon}

\usepackage{tikz}
\usetikzlibrary{arrows.meta, shapes.symbols, positioning,shadings,shadows}
\usetikzlibrary{shapes.network}

\usepackage{tcolorbox}
\tcbuselibrary{minted}
\usepackage{minted}

\begin{document}
\newminted{latex}{}
\maketitle

\begin{tikzpicture}[>=Latex,o/.tip={Rectangle}]
	\draw (-4,01) node[draw,router,fill=blue!20,arrow fill=blue,inner sep=0pt] (r){R};
	\draw (0,0) node[draw,switch,fill=green!20] (s) {$P_2$R};
	\draw (4,1) node[draw,hub,arrow fill=green] (h) {Hub};
	\node[draw,circle] at (0,2) (q) {PR};
	\draw[{Rectangle[red]}-{Rectangle[blue]}]   (s) --
		node[sloped,at start, above left,node font=\scriptsize,red]{192.168.0.1/24}
		node[sloped,at end, below right, node font=\scriptsize,blue]{192.168.0.254/24}
		(r);
	\draw[blue,->]  (q) -- (r);
	\draw[green] (s) -- (q);
	\draw[{Rectangle}-{Rectangle}] (h) -- (s);

	\draw (4,-4) node[draw,server] (www) {www};
	\draw[{Rectangle}-{Rectangle}] (www) -- (s);

	\draw (4,-2) node[draw,server] (dns) {$\mbox{DNS}_1$};
	\draw[{Rectangle}-{Rectangle}] (dns) -- (s);

	\draw (0,-2) node[draw,server] (nfs) {Network File Server (NFS)};
	\draw[{Rectangle}-{Rectangle}] (nfs) -- (s);

	\draw (5.5,-2) node[draw,server] (s1) {};
	\draw[{Rectangle}-{Rectangle}] (s1) -- (dns);

	\draw (nfs) -- (dns);

	\draw (-4,-1) node[draw,pc] (pc) {PC};
	\draw[o-o] (pc) -- (s);
	\draw (pc) |- (www);
\end{tikzpicture}



\begin{abstract}
	A TikZ shape library for drawing network diagrams.
\end{abstract}

\section{Installation}
There should be  four files supplied:
\begin{enumerate}
	\item \texttt{shapes.network.tex}  --- The documentation and guide
	\item \texttt{shapes.network.dtx}  --- The source for the library
	\item \texttt{shapes.network.ins}  --- The docstrip batch file for
		generating the library
	\item \texttt{Makefile}  --- A makefile for automating building and
		installation (installs to \verb:$HOME/texmf/tex:)
\end{enumerate}

%\tikzset{inner sep=0pt}
\section{Usage}
The library loads as any other shape library .
\begin{latexcode}
\usetikzlibrary{shapes.network}
\end{latexcode}

The shapes can then be used as node shapes in the usual fashion
\begin{latexcode}
\node[draw,router] (r1) {$R_1$};
\end{latexcode}
\begin{tikzpicture}
\node[draw,router] (r1) {$R_1$};
\end{tikzpicture}

The \verb:fill: option is used to change the colour of the device.
The colour of the arrows on the top of the device is given by the option
\verb:arrow fill:
\begin{latexcode}
\node[draw,router,fill=blue!50,arrow fill=blue!50!black]{};
\end{latexcode}

\begin{tikzpicture}
\node[draw,router,fill=blue!50,arrow fill=blue!50!black]{};
\end{tikzpicture}

Labels are displayed on the symbols, which try to adapt to fit the given text.
I have found that it is best to set \verb:inner sep: to zero.\marginpar{The
size adjustment \emph{needs} work!} for the router.
\begin{latexcode}
\node[draw,router,fill=blue!40,inner sep=0pt] {$R_1$};
\end{latexcode}
\begin{tikzpicture}
\node[draw,router,fill=blue!40,inner sep=0pt] {$R_1$};
\end{tikzpicture}

The set of shapes is shown in table \ref{shapes.list}

\begin{table}
\begin{tabular}{lll}
Router & \mintinline{latex}{\node[draw,router,fill=blue!35]{R1};} 
       & \rule[-12pt]{0pt}{32pt}
	   \tikz[baseline=0pt]\node[draw,router,fill=blue!35,inner sep=0pt]{R1}; \\
Switch & \mintinline{latex}{\node[draw,switch,fill=blue!35]{S1};} 
       & \rule[-12pt]{0pt}{32pt} \tikz[baseline=0pt]\node[draw,switch,fill=blue!35]{S1}; \\
Hub    & \mintinline{latex}{\node[draw,hub,fill=blue!35]{H1};} 
       & \rule[-12pt]{0pt}{32pt} \tikz[baseline=0pt]\node[draw,hub,fill=blue!35]{H1}; \\
\hline
Server & \mintinline{latex}{\node[draw,server,fill=yellow!35]{S1};} 
       & \rule[-12pt]{0pt}{32pt} \tikz[baseline=0pt]\node[draw,server,fill=yellow!35]{S1}; \\
PC     & \mintinline{latex}{\node[draw,pc,fill=yellow!35]{PC};} 
       & \rule[-12pt]{0pt}{32pt} \tikz[baseline=0pt]\node[draw,pc,fill=yellow!35]{PC}; \\
\end{tabular}
\caption{Network device shapes}
\label{shapes.list}
\end{table}

\clearpage
\section{Network diagrams}
Diagrams showing network configurations can be drawn using TikZ nodes and
positioning.


For example:
\begin{latexcode}
\begin{tikzpicture}
\node[draw,cloud,aspect=2.5,ball color=red!50,drop shadow] (net) {The Internet};
\node[draw,router,fill=blue!30,below=of net,label=below:Corporate ISP] (r) {$R_1$};
\node[draw,switch,fill=blue!30,right=of r] (sw) {$S_1$};
\node[draw,server,fill=yellow!35,right=of sw] (www) {www};
\node[draw,server,fill=yellow!35,right=of www] (sql) {sql};
\node[draw,server,fill=yellow!35,below=of sql] (nfs) {nfs};

\draw (net) -- (r) -- (sw) -- (www);
\draw (www) -- (sql);
\draw (www) -- (nfs);

\end{tikzpicture}
\end{latexcode}
\begin{tikzpicture}
\node[draw,cloud,aspect=2.5,ball color=red!50,drop shadow] (net) {The Internet};
\node[draw,router,fill=blue!30,below=of net,label=below:Corporate ISP] (r) {$R_1$};
\node[draw,switch,fill=blue!30,right=of r] (sw) {$S_1$};
\node[draw,server,fill=yellow!35,right=of sw] (www) {www};
\node[draw,server,fill=yellow!35,right=of www] (sql) {sql};
\node[draw,server,fill=yellow!35,below=of sql] (nfs) {nfs};

\draw (net) -- (r) -- (sw) -- (www);
\draw (www) -- (sql);
\draw (www) -- (nfs);

\end{tikzpicture}

Using the ability to position nodes on edges, we can label the addresses of
ports.

\begin{latexcode}
\begin{tikzpicture}
\node[draw,router,fill=blue!30] (R1) {};
\node[draw,switch,fill=blue!30,right=2in of R1] (sw) {};
\draw (R1) --(sw) 
	node[at start, font=\scriptsize, anchor=south west ] {192.168.1.254/24}
	node[at end, font=\scriptsize, anchor=north east ] {192.168.1.254/24};
\end{tikzpicture}
\end{latexcode}

\begin{tikzpicture}
\node[draw,router,fill=blue!30] (R1) {};
\node[draw,switch,fill=blue!30,right=2in of R1] (sw) {};
\draw (R1) --(sw) 
	node[at start, font=\scriptsize, anchor=south west ] {192.168.1.254/24}
	node[at end, font=\scriptsize, anchor=north east ] {192.168.1.254/24};
\end{tikzpicture}

\section{The Shapes}

The shapes and their defined anchors are shown below.

\tikzset{
shape example/.style= {color = black!30,
draw,
fill = yellow!30,
line width = .3mm,
inner xsep = 2cm,
inner ysep = .5cm}
}

\subsection{Router}
\begin{tikzpicture}
	\node[name=r, shape=router, shape example, node font=\Huge]{Router};
	\foreach \anchor/\placement in
        {center/above, base/below, text/left,
		 north/above, south/below, east/right, west/left,
         south west/left, south east/right,
	     north west/left, north east/right}
        \draw[shift=(r.\anchor)] plot[mark=x] coordinates{(0,0)}
                node[\placement] {\scriptsize\texttt{(.\anchor)}};
\end{tikzpicture}

\subsection{Switch}
\begin{tikzpicture}
	\node[name=r, shape=switch, shape example, node font=\Huge]{Switch};
	\foreach \anchor/\placement in
        {center/above, base/below, text/left,
		 north/above, south/below, east/right, west/left,
         south west/left, south east/right,
	     north west/left, north east/right}
        \draw[shift=(r.\anchor)] plot[mark=x] coordinates{(0,0)}
                node[\placement] {\scriptsize\texttt{(.\anchor)}};
\end{tikzpicture}

\subsection{Hub}
\begin{tikzpicture}
	\node[name=r, shape=hub, shape example, node font=\Huge]{Hub};
	\foreach \anchor/\placement in
        {center/above, base/below, text/left,
		 north/above, south/below, east/right, west/left,
         south west/left, south east/right,
	     north west/left, north east/right}
        \draw[shift=(r.\anchor)] plot[mark=x] coordinates{(0,0)}
                node[\placement] {\scriptsize\texttt{(.\anchor)}};
\end{tikzpicture}

\subsection{Server}
\begin{tikzpicture}
	\node[name=r, shape=server, shape example, node font=\Huge]{Server};
	\foreach \anchor/\placement in
        {center/above, base/below, text/left,
		 north/above, south/below, east/right, west/left,
         south west/left, south east/right,
	     north west/left, north east/right}
        \draw[shift=(r.\anchor)] plot[mark=x] coordinates{(0,0)}
                node[\placement] {\scriptsize\texttt{(.\anchor)}};
\end{tikzpicture}

\subsection{PC}
\begin{tikzpicture}
	\node[name=r, shape=pc, shape example, node font=\Huge]{PC};
	\foreach \anchor/\placement in
        {center/above, base/below, text/left,
		 north/above, south/below, east/right, west/left,
         south west/left, south east/right,
	     north west/left, north east/right}
        \draw[shift=(r.\anchor)] plot[mark=x] coordinates{(0,0)}
                node[\placement] {\scriptsize\texttt{(.\anchor)}};
\end{tikzpicture}

\appendix

\begin{quotation}
	\begin{itshape}
	The code is a little rough and ready, I still have to work on edge cases
	and aspect ratios.
\end{itshape}
\end{quotation}

\DocInput{shapes.network.dtx}

\end{document}

